\documentclass[twocolumn]{article}
\usepackage[hmargin={0.5in, 0.5in}, vmargin={0.8in, 0.8in}]{geometry}
\usepackage{booktabs}
\usepackage[table]{xcolor}
\usepackage{hyperref}
\usepackage[anythingbreaks]{breakurl}

\usepackage[compact]{titlesec}

\newcommand{\pkg}[1]{{\normalfont\fontseries{b}\selectfont #1}}
\let\proglang=\textsf
\let\code=\texttt

\usepackage{color}
\newcommand{\jy}[1]{\textcolor{red}{Updated: (#1)}}

\usepackage{enumitem}
\setlist{parsep=0pt, leftmargin=4mm, topsep=0pt, itemsep=8pt}
% \setlist{nolistsep}

\usepackage{fancyhdr}
\pagestyle{fancy}
\renewcommand{\headrulewidth}{0.4pt}
\renewcommand{\footrulewidth}{0.4pt}


\lhead{\sf STAT/BIST 5255}
\rhead{Fall 2021}
\lfoot{\sf jun.yan@uconn.edu}
\rfoot{\url{http://www.stat.uconn.edu/~jyan/}}
% \rhead{{\color{gray}\it Course Syllabus --- Spring 2021, \thepage/\pageref{LastPage}}}

\begin{document}
% \maketitle

\title{STAT/BIST 5255: Introduction to Data Science}
\date{August 31, 2021}

\maketitle

\thispagestyle{fancy}

\begin{description}
\item[Instructor:]
  Jun Yan\\
  % Austin 328\\
  % 860/486-3416\\
  jun.yan@uconn.edu

\item[Lectures:] 
  TuTh 3:30--4:45 pm @ AUST 313
  
  % \jy{Log into HuskyCT; Go into the course; Click on WebEx on the left menu and
  %   then you will see the lecture meetings. Recordings will be here too when
  %   they are available.}

\item[Office Hours:] 
  W 12:30 -- 1:30 pm @ WebEx virtual office
  \url{https://uconn-cmr.webex.com/meet/juy07002}
  or by appointment.

% \item[Grader:] Jing Li (\href{mailto:
%   jingli123@uconn.edu}{jingli123@uconn.edu})
  
\item[Prerequisites]
  Open to graduate students in Statistics, others with permission. Not
  open for credit to students who have passed STAT 3255.
    
\item[Course Description:]
  Introduction to data science for effectively storing, processing,
  visualizing, analyzing and making inferences from data to enable
  decision making. Topics include project management, data
  preparation, data visualization, statistical modeling, machine
  learning, distributed computing and ethics.
  
\item[Course Materials:]
  Lecture notes, assignments, sample code, datasets,  and other course material will be posted
  on the HuskyCT course website (available at \url{https://lms.uconn.edu/}).
  Please visit this site often to ensure timely obtainment of materials.
  
%{\red The lecture notes will be available online before each class.}
\item[Recommended Textbooks:]\hspace{0pt}
  \begin{enumerate}
  \item 
    ``Python Data Science Handbook: Essential Tools for Working with
    Data,'' First Edition, by Jake VanderPlas, O’Reilly Media, 2016.
 
  \item
    ``Python for Data Analysis: Data Wrangling with Pandas, NumPy, and
    IPython.''  Second Edition by Wes McKinney, O’Reilly Media, 2018.
  \item
    \href{https://www.practicaldatascience.org/html/not_a_mids_student.html}{Practical
    Data Science at Duke}.
  \end{enumerate}

\item[Computing:]
  \proglang{Python} will be the primary computing environment. You are
  free to choose any interactive development environment (Jupyter
  Notebook, VS Code, Emacs, etc.). Checkout
  \href{https://google.github.io/styleguide/pyguide.html}{Google
  Python Style Guide} and practice the recommendations.

  Git will be used for project management and work flows, and GitHub
  Classroom will be used for homework submission and grading. Learn
  and practice \href{https://gitexercises.fracz.com}{online}.

  Command line operation is essential. Point-and-clicks are enemies or
  replicability and automation. Pick the basics from any tutorial
  (e.g.,
  \href{https://ubuntu.com/tutorials/command-line-for-beginners}{Ubuntu}).

  Always be ready to learn new things.

\item[Grading:]
The grade for this course will be based on:
\begin{center}
  \rowcolors{2}{blue!20}{gray!20}
  \begin{tabular}{lr}
    \toprule
    Category                & Percentage\\
    \midrule
    Participation           &  10\% \\
    Homework              &  15\% \\
    Presentations          &  25\% \\
    Midterm Project      &  20\% \\
    Final Project            & 30\% \\
    \bottomrule
  \end{tabular}
\end{center} 

\item[Participation:]  Participation in the remote setting includes
  active use of HuskyCT Discussion Boards, answering questions posed
  during lecture, and evaluation of presentations from the
  undergraduate students.

\item[Homework:]\hspace{0pt}
\begin{itemize}
\item Homeworks will be assigned roughly weekly throughout the
  semester.  Students may consult amongst themselves or with the
  instructor, but each student must submit his/her own work.
  
\item All completed assignments are to be submitted by the due date. 
Assignments will be accepted up to 48 hours late, but with penalty.  If the
submission is within 48 hours of the due date and time, total amount of credit
available will be a linear function of time-beyond-due, ranging from 95\%--50\%
of the total points. Submissions over 48 hours late will not be graded and will
receive no credit.
		
\item No credit will be given for submitted assignments exhibiting
  duplication or copying of solutions (from peers or existing
  solutions).
\end{itemize}

\item[Midterm Project:]
  The midterm project will be an assigned one. It requires completion
  of the analysis component of a full data science project
  and presenting results.
  
\item[Project:] The final project will be due by 11:59pm, Friday,
  December 10, 2021. The topic will be of your choice. It should show
  the whole cycle of a real data science project. Consider projects at
  data science competition sites (e.g., \url{https://kaggle.com}).


\item[COVID Guidelines]
  See \url{https://covid.uconn.edu/campus-guidelines}.

\item[University Policies and Academic Integrity]
  See \url{http://provost.uconn.edu/syllabi-references}.
  
\end{description}
% \label{LastPage}
\end{document}
